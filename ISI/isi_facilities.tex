
\documentclass[11pt]{article}

\usepackage[margin=1in]{geometry}


\begin{document}


\section{Facilities at USC/ISI}

The University of Southern California's \emph{Information Sciences Institute}
(USC/ISI) is a unique academic research center with extensive
experience blending basic and applied research through
exploratory system development. USC/ISI has a distinguished history of
producing exceptional research contributions and successful prototype
systems funded by a variety of government agencies.

USC/ISI's \emph{Networking and Cybersecurity Division} (NCD) has a 40 year history of
leadership in network research.  It played major roles in such areas
as specifying the basic Internet protocol suite, developing the Domain
Name System, research on networked multimedia, integrated services,
network security, active networks,
virtual networks, sensor networks, Internet topology, optical
networks, and space networking.
%
ISI is also part of USC's Viterbi School of Engineering,
and ISI staff frequently teach and advise students in the
Computer Science and Electrical Engineering departments.
% USC/ISI is a component of USC's Viterbi School of Engineering, and
% maintains a close relationship with USC's departments of Computer
% Science and Electrical Engineering, as well as other USC departments
% and Los Angeles area universities.  Our staff frequently teach
% advanced courses at USC; a number of staff hold permanent USC academic
% appointments.  Our graduate students are drawn primarily from the USC
% CS and EE programs, with other departments and local universities as
% well.

ISI's NCD operates a number of laboratories.
The \emph{Postel Center for Experimental Networking}
  provides laboratory space for visiting students and scholars.
\emph{DETERlab} is the first
site in the Cyber Defense Technology Experimental Research testbed,
with a 691-node testbed for security research with a programmable
router fabric, with a deployment spanning USC/ISI, USC's main campus, and UC Berkeley,
  including support for Software-Defined Networking experiments.
%I-LENSE, the ISI Laboratory for Embedded Networked Sensor
%Experimentation, with hundreds of 8- and 32-bit embedded sensor nodes,
%and a multi-site underwater
%acoustic testbed in Marina del Rey;
The \emph{ANT Lab} (Analysis of Networking Traffic)
  carries out 24x7 monitoring of the IPv4 Internet for outages
  and supports over a petabyte of datasets with offsite backup.
We also operate
  a 440-core/220\,TB Hadoop cluster   % stats as of 2021-10-31
  allowing big-data processing with hardware-level control,
  and 1.6\,PB of RAID storage with off-site backup.
The ANT lab shares research data:
  as of Dec.~2020, we have distributed 2194 datasets 
  (402\,TB of data compressed, % 402381825428890
  1.3\,PB if uncompressed) % 1302194383298010
    to 452 researchers in 327 different organizations.

\iftrue
USC operates \emph{b.root-servers.net}, one of the 13 root DNS nameservers, as a service
  for the Internet.
Started when DNS was invented at USC/ISI,
  USC/ISI's root name server is currently operated collaboratively by the USC/ISI networks
  division and USC's Information Technology Services
and is hosted at USC and 5 other sites spread geographically around
  the globe.
All sites use a cluster of computers for performance and robustness.
Our operation includes integrated measurement facilities in support of operation,
  and a dedicated 192-core % 2021-01-23
  Hadoop cluster.
While recognizing the server's top priority is providing an operational service for the Internet,
  \emph{b.root-servers.net} also serves as research infrastructure, supporting
  safe and privacy-sensitive experimentations and analysis.
\fi

\iffalse
For very large computations,
  we work closely with
  USC's \emph{High Performance Computing and Communications Facility} (HPCC), the 5th fastest academic
computer in the U.S. with more than 2500 2- or 4-core compute
  nodes and over 400TB of disk.
%We have a long-standing arrangement with HPCC to
%  use their computers for LANDER-specific processing, and to
%  have them host LANDER-specific data capture and storage equipment.
Inside HPCC we operate a small
  8-node Hadoop cluster dedicated for dataset processing.
\fi

USC/ISI has rich Internet connectivity,
  with OC-48c (2.4 Gb/s) connectivity between sites in L.A. and Virginia,
  dark fiber and 10~Gb/s connectivity to USC's downtown campus,
  and also Internet peerings
  through CALREN2 and Internet2/Abilene.
%
In addition,
USC's Distance Education Network supports
remote education  with multiple studios for class material's broadcast and
webcast.

% External collaborators (\autoref{sec:letters}) each have separate
% deployed anycast-based infrastructure networks that they will use in
% their application of our approaches in their networks.

%  USC supports remote lectures and workshops using the access
%grid, supporting large-scale remote conferencing facilities that will
%support better integrated collaboration of participating organization
%through lectures and meetings, and open participation in center
%supported workshops, especially among traditionally disenfranchised
%educational institutions. Multimedia Laboratory includes plasma
%displays, DV camera and storage facilities, audio mixers and echo
%cancellers, and at the East coast Arlington campus, a High Definition
%Television production and transmission capability.

\end{document}
